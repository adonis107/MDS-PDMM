\documentclass{beamer}
\usetheme{Madrid}
\usecolortheme{default}
\useoutertheme{split}
\useinnertheme{circles}
\usepackage{xcolor}
\usepackage{mathrsfs}
\usepackage{amsmath}
\usepackage{amssymb}
\usepackage{booktabs}
\usepackage{hyperref}
\usepackage{animate}
\usepackage{caption}
\captionsetup[figure]{font=tiny}

\usepackage[
    backend=biber,
    style=alphabetic,
    sorting=ynt
]{biblatex}
\addbibresource{bibliography.bib}

% --- Couleurs CS ---
\definecolor{CSred}{RGB}{160,32,60}
\definecolor{CSgrey}{RGB}{136,121,150}
\definecolor{UPSred}{RGB}{97,21,58}
\setbeamercolor{titlelike}{bg=CSred}
\setbeamerfont{title}{series=\bfseries}
\setbeamercolor{palette primary}{bg=CSred,fg=white}
\setbeamercolor{palette secondary}{bg=CSred,fg=white}
\setbeamercolor{palette tertiary}{bg=CSgrey,fg=white}
\setbeamercolor{palette quaternary}{bg=CSgrey,fg=white}
\setbeamercolor{structure}{fg=CSgrey}

\title[Détection de Manipulation]{Détection Probabiliste de Manipulation de Marchés}
\subtitle{Soutenance Intermédiaire - Projet de Recherche}
\author[Jamal, Martini]{Adonis~Jamal \and Jean Martini\\Encadrant : Damien Challet}
\institute[CentraleSupélec]{CentraleSupélec -- Laboratoire MICS}
\date[4 décembre 2025]{4 décembre 2025}

\titlegraphic{
    \includegraphics[height=0.8cm]{img/logo_cs}
    \hspace{2cm}
    \includegraphics[height=0.8cm]{img/logoMICS.png}
}

\begin{document}

\frame{\titlepage}

\begin{frame}
    \frametitle{Sommaire}
    \tableofcontents
\end{frame}

% ==============================================================================
% SECTION 1: INTRODUCTION
% ==============================================================================
\section{Introduction}
\subsection{Contexte}

\begin{frame}{Le carnet d'ordres (LOB)}
    \frametitle{Les marchés et le Carnet d'Ordres}
    \begin{itemize}
        \item \textbf{Limit Order Book (LOB) :} Répertoire central des intentions d'achat (Bid) et de vente (Ask).
        \item \textbf{Dynamique :} Le prix médian (\textit{mid-price}) réagit à l'état instantané et au flux d'ordres passé (Order Flow).
        \item \textbf{Données :} 
        \begin{itemize}
            \item Données Haute Fréquence (Level 3 / Market-by-order).
            \item Nécessité de traiter des millions d'événements par jour.
        \end{itemize}
    \end{itemize}
    \begin{block}{Problématique}
        La microstructure du marché peut être exploitée pour induire les autres participants en erreur.
    \end{block}
\end{frame}

\begin{frame}{Le spoofing}
    \frametitle{Le spoofing : une manipulation structurelle}
    \begin{itemize}
        \item \textbf{Définition :} Placer des ordres non-bona fide (sans intention d'exécution) pour créer une fausse impression de liquidité ou de pression directionnelle.
        \item \textbf{Tactiques courantes :}
        \begin{itemize}
            \item \textit{Layering :} Empilement d'ordres à différents niveaux de prix.
            \item \textit{Vacuuming :} Création de vides de liquidité pour provoquer des mouvements brusques.
        \end{itemize}
        \item \textbf{Mécanisme :} 
        \begin{enumerate}
            \item Placer un gros ordre d'achat (Fake) loin du prix.
            \item Attendre que le prix monte (réaction des algos).
            \item Vendre réellement (Bona fide) à un prix plus élevé.
            \item Annuler l'ordre d'achat initial.
        \end{enumerate}
    \end{itemize}
\end{frame}

\subsection{Hypothèses}
\begin{frame}{Hypothèses de travail}
    \begin{block}{Hypothèses Microstructurelles}
        \begin{itemize}
            \item La distance de placement des ordres limites joue un rôle critique dans la formation des prix (contrairement aux modèles basés uniquement sur le volume au best bid/ask) \cite{tao_2020}.
            \item Un agent manipulateur agit rationnellement pour minimiser son coût d'exécution espéré.
        \end{itemize}
    \end{block}

    \begin{block}{Hypothèse Comparative}
        En l'absence de régulation stricte, les carnets d'ordres de crypto-monnaies (CEX) présentent :
        \begin{itemize}
            \item Une fréquence de spoofing plus élevée.
            \item Une sensibilité (impact prix) plus forte aux ordres de spoofing que les marchés actions traditionnels.
        \end{itemize}
    \end{block}
\end{frame}

\subsection{Objectifs}
\begin{frame}{Objectifs du projet}
    Le projet vise à développer une meilleure caractérisation de la \textbf{"Spoofability"} des marchés.
    \vspace{0.3cm}

    \textbf{Objectifs principaux :}
    \begin{enumerate}
        \item \textbf{Implémentation de modèles et features :} Réseaux de neurones \cite{fabre_2025,poutre_2024,tao_2020}, exploration de l'état de l'art.
        \item \textbf{Comparaison inter-marchés :} 
        \begin{itemize}
            \item Marchés Actions (Régulés, surveillés).
            \item Marchés Crypto (Non régulés, suspicion de forte manipulation).
        \end{itemize}
        \item \textbf{Validation de l'hypothèse :} Les marchés actions sont-ils vraiment moins sensibles (spoofables) que les marchés crypto ? Arrivons-nous à détecter ces manipulations ? 
    \end{enumerate}
\end{frame}

\subsection{Avancement et Plan}
\begin{frame}{Avancement actuel et plan de travail}
    \textbf{Réalisé (Mois 1) :}
    \begin{itemize}
        \item Étude bibliographique approfondie (Fabre, Tao, Poutré).
        \item Prise en main des données (L3 fournies par l'encadrant).
        \item Architecture du code Python structurée (pré-traitement, feature engineering, modèles).
    \end{itemize}

    \textbf{Planification (Mois 2-5) :}
    \begin{itemize}
        \item \textbf{Décembre - Janvier :} Implémentation complète et calibration des modèles sur les données. Exploration des features.
        \item \textbf{Février :} Analyse de sensibilité. Quantification de la spoofability comparée.
        \item \textbf{Mars :} Exploration avancée (anomalies, clustering, autres modèles).
        \item \textbf{Avril :} Rédaction du rapport final et préparation soutenance.
    \end{itemize}
\end{frame}

% ==============================================================================
% SECTION 2: ÉTAT DE L'ART (BIBLIO)
% ==============================================================================
\section{État de l'art}
\begin{frame}{Synthèse Bibliographique}
    Notre approche s'appuie sur trois axes de recherche récents :
    
    \begin{itemize}
        \item \textbf{Approche Probabiliste \& Features Hawkes} \cite{fabre_2025} :
        Utilisation de réseaux de neurones pour prédire non pas un prix, mais les paramètres d'une distribution de prix ($\mu, \sigma, \alpha$), en utilisant des variables de flux d'ordres multi-échelles (temps, volume, distance).
        
        \item \textbf{Détection Non-Supervisée (Deep Learning)} \cite{poutre_2024} :
        Utilisation d'Auto-encodeurs (Transformers) pour apprendre la "normalité" du marché et détecter les anomalies via la distance de reconstruction ou SVM One-Class.
        
        \item \textbf{Modélisation de l'Imbalance} \cite{tao_2020} :
        Étude théorique de l'impact de l'imbalance multi-niveaux sur le mouvement des prix et stratégies optimales de spoofing.
    \end{itemize}
\end{frame}

% ==============================================================================
% SECTION 3: MÉTHODOLOGIE
% ==============================================================================
\section{Méthodologie}

\subsection{Finance Quantitative}
\begin{frame}{Détection par la "Spoofability"}
    \frametitle{Critère de Détection Financier}
    Nous adoptons le point de vue du manipulateur rationnel.
    
    \begin{block}{Fonction de Coût Espéré $\mathbb{E}[C]$}
        Le spoofer compare deux scénarios pour exécuter un ordre cible $q$ :
        \begin{enumerate}
            \item \textbf{Exécution normale :} Coût d'exécution sans intervention.
            \item \textbf{Avec Spoofing :} Coût d'exécution après insertion d'un faux ordre $Q$ à distance $\delta$.
        \end{enumerate}
    \end{block}
    
    \begin{block}{Règle de Détection}
        Un ordre $(Q, \delta)$ est suspect si son insertion améliore significativement le coût d'exécution espéré sur le côté opposé :
        $$ \Delta \mathcal{C}(Q, \delta) = \mathbb{E}[C_{\text{normal}}] - \mathbb{E}[C_{\text{spoof}}] > 0 $$
    \end{block}
\end{frame}

\subsection{Machine Learning}
\begin{frame}{Architecture du Modèle (PNN)}
    \frametitle{Réseau de Neurones Probabiliste}
    L'objectif n'est pas de classifier (Spoof/Pas Spoof) mais d'apprendre la dynamique des prix $P(\Delta p | \mathcal{F}_t)$.
    
    \begin{itemize}
        \item \textbf{Entrée $x$ :} Vecteur de features (Spread, Flux d'ordres Limites multi-échelles, Flux d'ordres Marché).
        \item \textbf{Modèle :} Feed-forward Neural Network (MLP) léger pour le temps réel.
        \item \textbf{Sortie $\Theta(x)$ :} Paramètres d'une distribution (ex: Gaussienne Asymétrique).
        $$ \Delta p \sim \mathcal{SN}(\mu(x), \sigma(x), \alpha(x)) $$
        \item \textbf{Entraînement :} Maximisation de la vraisemblance (Maximum Likelihood Estimation) sur données historiques saines.
    \end{itemize}

    Le modèle PNN sert à calculer les espérances en prédisant le mouvement de prix $\Delta p$ induit.
\end{frame}

% ==============================================================================
% SECTION 4: CONCLUSION
% ==============================================================================
\section{}
\begin{frame}{}
    \centering
    \LARGE{Merci pour votre attention}
\end{frame}


% ==============================================================================
% SECTION : BIBLIOGRAPHIE
% ==============================================================================
\begin{frame}[allowframebreaks]{Bibliographie}
    \frametitle{Bibliographie}
    \nocite{*}
    \printbibliography
\end{frame}

% ==============================================================================
% SECTION : ANNEXE
% ==============================================================================
\appendix
\section{Annexe}
\subsection{Organisation du Travail}
\begin{frame}{Organisation du travail d'équipe}
    \begin{block}{Organisation et dispositif}
        \begin{itemize}
            \item Réunions hebdomadaires avec l'encadrant (D. Challet).
            \item Gestion de projet agile : ré-évaluation périodique des objectifs.
            \item Outils : Teams (Comms), Git (Versionning Code), Overleaf (Rapport).
        \end{itemize}  
    \end{block}
    \begin{block}{Répartition du travail}
        \begin{itemize}
            \item \textbf{Adonis :} Code, travail sur les méthodes, orienté finance quantitative. 
            \item \textbf{Jean :} Code, travail sur les méthodes, orienté machine learning.
        \end{itemize}
    \end{block}
\end{frame}

\subsection{Feature Engineering}
\begin{frame}{Feature Engineering : Processus de Hawkes}
    \frametitle{Construction des variables explicatives}
    Pour capturer la dynamique temporelle et spatiale du carnet, nous utilisons des variables inspirées des processus de Hawkes marqués.
    
    Soit $N_t$ le processus de comptage des ordres. On définit les variables de flux $L_t$ (Limit) et $M_t$ (Market) :
    
    \begin{equation}
        L_t(\phi) = \int_{(-\infty, t]} \phi(t-s, v_s, \delta_s) dN_s
    \end{equation}
    
    \begin{itemize}
        \item \textbf{Noyau $\phi$ :} Intègre trois dimensions cruciales :
        \begin{itemize}
            \item \textbf{\textit{Temps :}} Décroissance exponentielle $e^{-\beta t}$ (mémoire du marché).
            \item \textbf{\textit{Volume ($v_s$) :}} Impact de la taille de l'ordre.
            \item \textbf{\textit{Distance ($\delta_s$) :}} Distance au mid-price (innovation majeure par rapport à l'imbalance classique).
        \end{itemize}
        \item \textbf{Multi-échelles :} Combinaison de plusieurs horizons temporels ($\beta$) et échelles de distance ($\eta$).
    \end{itemize}
\end{frame}



\end{document}